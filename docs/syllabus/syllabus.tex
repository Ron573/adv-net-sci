\documentclass[11pt, a4paper]{article}
%\usepackage{geometry}
\usepackage[inner=1.5cm,outer=1.5cm,top=2.5cm,bottom=2.5cm]{geometry}
\pagestyle{empty}
\usepackage{graphicx}
\usepackage{fancyhdr, lastpage, bbding, pmboxdraw}
\usepackage[usenames,dvipsnames]{color}
\definecolor{darkblue}{rgb}{0,0,.6}
\definecolor{darkred}{rgb}{.7,0,0}
\definecolor{darkgreen}{rgb}{0,.6,0}
\definecolor{red}{rgb}{.98,0,0}
\usepackage[colorlinks,pagebackref,pdfusetitle,urlcolor=darkblue,citecolor=darkblue,linkcolor=darkred,bookmarksnumbered,plainpages=false]{hyperref}
\renewcommand{\thefootnote}{\fnsymbol{footnote}}

\pagestyle{fancyplain}
\fancyhf{}
\lhead{ \fancyplain{}{Advanced Topics in Network Science} }
%\chead{ \fancyplain{}{} }
\rhead{ \fancyplain{}{\today} }
%\rfoot{\fancyplain{}{page \thepage\ of \pageref{LastPage}}}
\fancyfoot[RO, LE] {page \thepage\ of \pageref{LastPage} }
\thispagestyle{plain}

%%%%%%%%%%%% LISTING %%%
\usepackage{listings}
\usepackage{caption}
\DeclareCaptionFont{white}{\color{white}}
\DeclareCaptionFormat{listing}{\colorbox{gray}{\parbox{\textwidth}{#1#2#3}}}
\captionsetup[lstlisting]{format=listing,labelfont=white,textfont=white}
\usepackage{verbatim} % used to display code
\usepackage{fancyvrb}
\usepackage{acronym}
\usepackage{amsthm}
\VerbatimFootnotes % Required, otherwise verbatim does not work in footnotes!



\definecolor{OliveGreen}{cmyk}{0.64,0,0.95,0.40}
\definecolor{CadetBlue}{cmyk}{0.62,0.57,0.23,0}
\definecolor{lightlightgray}{gray}{0.93}



\lstset{
%language=bash,                          % Code langugage
basicstyle=\ttfamily,                   % Code font, Examples: \footnotesize, \ttfamily
keywordstyle=\color{OliveGreen},        % Keywords font ('*' = uppercase)
commentstyle=\color{gray},              % Comments font
numbers=left,                           % Line nums position
numberstyle=\tiny,                      % Line-numbers fonts
stepnumber=1,                           % Step between two line-numbers
numbersep=5pt,                          % How far are line-numbers from code
backgroundcolor=\color{lightlightgray}, % Choose background color
frame=none,                             % A frame around the code
tabsize=2,                              % Default tab size
captionpos=t,                           % Caption-position = bottom
breaklines=true,                        % Automatic line breaking?
breakatwhitespace=false,                % Automatic breaks only at whitespace?
showspaces=false,                       % Dont make spaces visible
showtabs=false,                         % Dont make tabls visible
columns=flexible,                       % Column format
morekeywords={__global__, __device__},  % CUDA specific keywords
}

%%%%%%%%%%%%%%%%%%%%%%%%%%%%%%%%%%%%
\begin{document}
\begin{center}
{\Large \textsc{Advanced Topics in Network Science}}
\end{center}
\begin{center}
Fall 2025
\end{center}
%\date{September 26, 2014}

\begin{center}
\rule{6in}{0.4pt}
\begin{minipage}[t]{.75\textwidth}
\begin{tabular}{llcccll}
\textbf{Instructor:} & Sadamori Kojaku & & &  & \textbf{Time:} & W16:40 -- 19:40 \\
\textbf{Email:} &  \href{mailto:skojaku@binghamton.edu}{skojaku@binghamton.edu} & & & & \textbf{Place:} & G22 Engineering Bldg. \\
\textbf{Office:} &  J19 Engineering Bldg. & & & & \textbf{Credit:} & 3 \\
\textbf{Office Hour:} &  14:30-16:30 Tue \& Thur & & & & \textbf{Zoom:} & \href{https://binghamton.zoom.us/my/skojaku.zoom}{my/skojaku.zoom}
\end{tabular}
\end{minipage}
\rule{6in}{0.4pt}
\end{center}
\vspace{.5cm}
\setlength{\unitlength}{1in}
\renewcommand{\arraystretch}{2}

\noindent\textbf{Course Pages:} \begin{enumerate}
\item Course Website \url{https://skojaku.github.io/adv-net-sci}
\item Course Github \url{http://github.com/skojaku/adv-net-sci}
\end{enumerate}

\vskip.15in
\noindent\textbf{Office Hours:} Friday 10:00 - 14:00.

\vskip.15in
\noindent\textbf{Main References:} %\footnotemark
Below is a curated list of essential books that will be referenced during the course.
\begin{itemize}
\item Mark Newman. {\textit{Networks (Second Edition)}}. Oxford University Press, 2018.
\item Filippo Menczer, Santo Fortunato, and Clayton A. Davis. {\textit{A First Course in Network Science}}. Cambridge University Press, 2020.
\item James Bagrow and Yong-Yeol Ahn. {\textit{Working with Network Data: A Data Science Perspective}}. Cambridge University Press, 2024.
\end{itemize}

% \footnotetext{Downloadable ebook versions are available on AeLP.}

\vskip.15in
\noindent\textbf{Objectives:}
Networks are all around us, from the vast expanse of the Internet to the intricate web of social connections that we build in our daily lives. But networks aren't just limited to the human realm---they can be found in every corner of the natural world, such as the complex interactions between animals, proteins, viruses, and DNAs. In recent years, we have witnessed remarkable advances in Al/ML and an ever-increasing volume and quality of data. Together, they offer an unprecedented opportunity to unlock the secrets of the world around us. This course is an introduction to network data analysis from the bottom up: through the interactions with network data and tools, we will learn how to store, manipulate, compute, and leverage network data in practice, as well as their underlying theoretical foundations.


\vskip.15in
\noindent\textbf{Expected Student Learning Outcomes:}
After completing this course, students will:
\begin{itemize}
    \item be able to interpret and evaluate modern network science literature, concepts, methodologies, tools, and recent research topics,
    \item be able to conduct advanced network modeling, analysis and simulation using appropriate mathematical/computational means,
    \item be able to design and conduct original research using network science methods and tools, and
    \item be able to demonstrate integration of Systems Science and/or Industrial and Systems Engineering knowledge and techniques and advanced network modeling/analysis/simulation skills in the form of a final project.
\end{itemize}

\vskip.15in
\noindent\textbf{Prerequisites:}
SSIE-523. Fluency in Python. Basic understanding of mathematics and statistics.

\clearpage

%\vspace*{.25in}
\noindent \textbf{Course Outline:}
\begin{center}
\begin{minipage}{5in}
\begin{flushleft}
\\textbf{August 2025:} \\
08/21 (Thu) \dotfill Intro: Seven Bridges of Königsberg \\
08/26 (Tue) \dotfill Intro: Seven Bridges of Königsberg \\
08/28 (Thu) \dotfill Small-world Networks \\

\\textbf{September 2025:} \\
09/02 (Tue) \dotfill No class (Monday classes meet) \\
09/04 (Thu) \dotfill Small-world Networks (cont'd) \\
09/09 (Tue) \dotfill Network Robustness \\
09/11 (Thu) \dotfill Project Matchmaking \\
09/16 (Tue) \dotfill Friendship Paradox \\
09/18 (Thu) \dotfill Friendship Paradox \\
09/23 (Tue) \dotfill No class (Rosh Hashanah) \\
09/25 (Thu) \dotfill Community Detection I \\
09/30 (Tue) \dotfill Community Detection I \\

\\textbf{October 2025:} \\
10/02 (Thu) \dotfill No class (Yom Kippur) \\
10/07 (Tue) \dotfill Community Detection II \\
10/09 (Thu) \dotfill Community Detection II \\
10/14 (Tue) \dotfill Centrality \\
10/16 (Thu) \dotfill Centrality \\
10/21 (Tue) \dotfill Random walks \\
10/23 (Thu) \dotfill Random walks \\
10/28 (Tue) \dotfill Random graphs \\
10/30 (Thu) \dotfill Random graphs \\

\\textbf{November 2025:} \\
11/04 (Tue) \dotfill Network compression using graph spectra \\
11/06 (Thu) \dotfill Network compression using graph spectra \\
11/11 (Tue) \dotfill From connectivities to distance: Graph embeddings \\
11/13 (Thu) \dotfill From connectivities to distance: Graph embeddings \\
11/18 (Tue) \dotfill Graph Neural Networks \\
11/20 (Thu) \dotfill Graph Neural Networks \\
11/25 (Tue) \dotfill Course Project Work Time \\
11/27 (Thu) \dotfill No class (Thanksgiving) \\

\\textbf{December 2025:} \\
12/02 (Tue) \dotfill Presentation \\
12/04 (Thu) \dotfill Presentation \\
Final Examinations \dotfill ~ 12/08-12/12 \\
\end{minipage}
\end{center}

\vspace*{.15in}
\noindent\textbf{Grading Policy:}
\begin{itemize}
\item \textit{Grading Item}: Quiz (10\%), In-class Presentation (10\%), Assignments (20\%). Exam (30\%). Project (30\%).
\item \textit{Bonus points}: 30\% equivalent of the total grading points.
\begin{itemize}
    \item 10\% bonus for the best project (one team).
    \item 10\% for the best Network of the Week Presentation (one team).
    \item 10\% bonus for the best question-answer assignment (2\% for each assignment: no limit on the number of recipients).
\end{itemize}
\item \textit{Credits}: 3 credits.
\item \textit{Grading}: Normal grading; A through F.
\end{itemize}

\vskip.15in
\noindent\textbf{Course structure:}
``Don't think! Feeeeeel'' is a famous quote by Bruce Lee in the movie \textit{Enter the Dragon}, and this is the guiding philosophy of this course.
The primary goal of this course is to \textbf{feel} the concepts and tools of network science through pen-and-paper exercises and hands-on coding.
The course will first cover the practical skills of network science, followed by their theoretical foundations.
The course activities include:
\begin{itemize}
\item The class will begin with a weekly quiz on Brightspace about the lecture topics in the previous week. This quiz is intended to review the lecture topics and to test your understanding of the concepts and tools of network science.
\item Biweekly coding assignemnt on Github Classroom. The assignments are intended for practicing and equipping you with the skills to analyze network data.
\item A short lecture to cover the theoretical aspects.
\end{itemize}

\vskip.15in
\noindent\textbf{Communications:}
We use Discord for quicker informal communications, Q\&A, team discussions, and other casual conversations. We will send you an invitation link through Brightspace. Feel free to NOT use your full name (e.g., “Jane D.”)
Announcements will be sent via Brightspace and Discord. Many course-related information will be shared on Discord. So, you will miss a lot of information if you are not on Discord.

\vskip.15in
\noindent\textbf{Office Hours:}
\begin{itemize}
    \item \textbf{Sadamori Kojaku:} 14:30-16:30 Tuesdays and Thursdays at J19 Engineering Building (in person) or Zoom (\url{https://binghamton.zoom.us/my/skojaku.zoom}).
\end{itemize}

\clearpage
\noindent\textbf{Important Dates:}
\begin{center} \begin{minipage}{3.8in}
\begin{flushleft}
Project Proposal \#1      \dotfill ~09/30 \\
Project Final Paper \#2      \dotfill ~12/05 \\
Project Presentation \#3      \dotfill ~12/09 \\
Final Exam       \dotfill ~12/08-12/12 \\
\end{flushleft}
\end{minipage}
\end{center}

\vskip.15in
\noindent\textbf{Course Policy:}

\begin{itemize}
    \item \textbf{Attendance}: If you are not able to attend the class in person, please request an excuse over email one day before the class. We may not accept excuses for reasons other than illness, accidents, job interviews, and conference travels. If you are not able to attend in person for more than two weeks due to illness or some other legitimate reason, please request an acceptable accommodation over email.
    \item \textbf{Laptop and mobile}: We want to engage in-class activities with you together. Please refrain from using laptops and mobile phones unless instructed.
    \item \textbf{Credit hours}: This course is a 3-credit course, which means that in addition to the scheduled lectures/discussions, students are expected to do at least 6.5 hours of course-related work each week during the semester. This includes things like: completing assigned readings, participating in lab sessions, studying for tests and examinations, preparing written assignments, completing internship or clinical placement requirements, and other tasks that must be completed to earn credit in the course.
    \item \textbf{Generative AI}: You may use artificial intelligence tools as learning aids for understanding course materials. However, the final submitted assignment must be original work produced by the individual student alone. If parts of the assignments are produced by generative AIs, you must indicate the generated parts and cite the source AIs. Refer to this format guideline: \url{https://style.mla.org/citing-generative-ai/}
    \item \textbf{Data backup}: You have the responsibility of backing up all your data and code. Always back up your code and data. You should at least use Google Drive or Dropbox at the minimum. You can also use cloud services like Google Colaboratory. Ideally, learn version control systems and use \url{https://github.com}. Loss of data, code, or papers (e.g. due to malfunction of your laptop) is not an acceptable excuse for delayed or missing submission.
    \item \textbf{Graduate Academic Consultants}: Binghamton University's Graduate Academic Consultants can help you with projects in this course. This is a free service. \url{https://www.binghamton.edu/grad-school/academic-support/graduate-academic-consultants/}
    \item \textbf{Disabilities}: Every attempt will be made to accommodate qualified students with disabilities (e.g. mental health, learning, chronic health, physical, hearing, vision, neurological, etc.). You must have established your eligibility for support services through Services for Students with Disabilities. Note that services are confidential, may take time to put into place, and are not retroactive. The office is located in the University Union, room 119. Captions and alternate media for print materials may take three or more weeks to get produced. Please contact Disability Services for Students at \url{https://www.binghamton.edu/ssd/index.html} or 607-777-2686 as soon as possible if accommodations are needed.
    \item \textbf{Bias-based incidents}: Any act of discrimination or harassment based on race, ethnicity, religious affiliation, gender, gender identity, sexual orientation, or disability can be reported at \url{https://www.binghamton.edu/diversity-equity-inclusion/reportbias.html} or to the Binghamton University Affirmative Action Officer at 607-777-4775.
    Sexual misconduct and Title IX
    \item \textbf{Sexual misconduct and Title IX}: Title IX and BU’s Sexual Harassment Policy regard any form of sexual harassment as a violation of the standards of conduct required of all persons associated with the institution. If you have experienced sexual misconduct or know someone who has, you can ask support from the University Counseling Center at 607-777-2772 (counseling, advocacy, and advice services). It is also important that you know that Title IX and University policy require me to share any information brought to my attention about potential sexual misconduct with the campus Deputy Title IX Coordinator or BU’s Title IX Coordinator. In that event, those individuals will work to ensure that appropriate measures are taken and resources are made available. Protecting student privacy is of utmost concern, and information will only be shared with those that need to know to ensure the University can respond and assist. Visit \url{https://www.binghamton.edu/counseling/resources/faculty/assault.html} and \url{https://www.binghamton.edu/services/title-ix/index.html} to learn more.
    \item \textbf{Mental health issues}: If you have any mental health issues, don’t hesitate to contact BU’s \href{https://www.binghamton.edu/counseling/index.html}{University Counseling Center}, which provides free counseling sessions. Also, please contact Disability Services for Students at \href{https://www.binghamton.edu/ssd/index.html}{Services for Students with Disabilities} or 607-777-6893 as soon as possible if accommodations are needed.
    \item \textbf{Academic Integrity}: Academic integrity is fundamental to the mission of our university. All students are expected to uphold the highest standards of academic honesty in their coursework and research. Violations of academic integrity include plagiarism, cheating, unauthorized collaboration, fabrication, and misrepresentation. For full details on the Academic Honesty Code, procedures, and appeals process, please refer to \href{https://www.binghamton.edu/watson/about/academic-honesty.html}{the Student Academic Honesty Code}. If you are unsure about what constitutes academic dishonesty in any situation, ask your instructor for clarification.
\end{itemize}

%%%%%% THE END
\end{document}