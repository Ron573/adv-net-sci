\documentclass[tikz]{standalone}
% \usepackage{tikz} % already loaded by the documentclass


\begin{document}
%%| caption: Network Connections vs Geographic Distance - Why computational tools matter

\begin{tikzpicture}[scale=1.5]
  % Geographic view (left side)
  \node[rectangle, draw, fill=red!10, text width=4cm, align=center] at (-3, 3) {
    \textbf{Geographic Thinking}\\
    Disease spreads to nearby places first
  };
  
  % Cities with geographic layout
  \node[circle, draw, fill=blue!20] (mexico) at (-4, 1) {Mexico};
  \node[circle, draw, fill=blue!20] (usa) at (-3, 2) {USA};
  \node[circle, draw, fill=blue!20] (guatemala) at (-4.5, 0.5) {Guatemala};
  \node[circle, draw, fill=blue!20] (japan) at (-1, 1.5) {Japan};
  
  % Geographic distance arrows
  \draw[dashed, gray] (mexico) -- (usa);
  \draw[dashed, gray] (mexico) -- (guatemala);
  \draw[dashed, gray] (mexico) -- (japan);
  
  \node[red] at (-3, 0) {❌ Poor prediction};
  
  % Network view (right side)
  \node[rectangle, draw, fill=green!10, text width=4cm, align=center] at (3, 3) {
    \textbf{Network Thinking}\\
    Disease follows flight connections
  };
  
  % Same cities with network layout
  \node[circle, draw, fill=green!20] (mexico2) at (2, 1) {Mexico};
  \node[circle, draw, fill=green!20] (usa2) at (3, 2) {USA};
  \node[circle, draw, fill=green!20] (guatemala2) at (1.5, 0.5) {Guatemala};
  \node[circle, draw, fill=green!20] (japan2) at (4, 1.5) {Japan};
  
  % Flight connections (thicker = more flights)
  \draw[thick, blue] (mexico2) -- (usa2);
  \draw[very thick, blue] (mexico2) -- (japan2);
  \draw[thin, blue] (mexico2) -- (guatemala2);
  
  \node[green!70!black] at (3, 0) {✓ Accurate prediction};
  
  % Arrow between views
  \draw[->, ultra thick, purple] (0, 2) -- (0, 1.5);
  \node[purple, align=center] at (0, 0.8) {Computational\\analysis\\needed!};
\end{tikzpicture}
\end{document}
      